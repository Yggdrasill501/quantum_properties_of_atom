\documentclass{article}

\usepackage[slovak]{babel}
\usepackage[letterpaper,top=2cm,bottom=2cm,left=3cm,right=3cm,marginparwidth=1.75cm]{geometry}

% Useful packages
\usepackage{amsmath}
\usepackage{graphicx}
\usepackage[colorlinks=true, allcolors=blue]{hyperref}

\title{Meranie kvantových vlastností atómu}
\author{Rebecca Kozmová, Filip Žitný, Šimon Nemčič}

\begin{document}
\maketitle



\section{Pracovné úlohy}
1. Zostavte aparatúru pre prevedenie Franck-Hertzovho pokusu a nechajte ortuťovú trubicu zahriať na zvolenú teplotu.\\
2. Pozorujte závislosť prechádzajúceho prúdu $I_A$ na urýchlujúcom napätí $U_A$ (Franckovu-Hertzovu krivku) na digitálnom osciloskope. \\Nájdite optimálne hodnoty parametrov $U_1$ a $U_3$ a diskutujte ich vplyv na podobu Franck-Hertzovej krivky.\\
3. Pomocou dodatočných voltmetrov  premerajte Franck-Hertzovu krivku pre napätie $U_2$ v
rozmedzí  od 0 do 30 V a zostavte jej graf. Namerajte aspoň 80 hodnôt.\\
4. V okolí maxím a miním Franck-Hertzovej krivky preložte namerané dáta polynómami  druhého
stupňa a určte súradnice  extrémov. Postupnou metódou vypočítajte  excitačnú  energiu atómu ortute. Výsledok porovnejte s tabuľkovou hodnotou.\\
5. Využite výslednú  excitačnú energiu a hodnotu Planckovej konštanty h získanú v úlohe 11a a s ich pomocou spočítajte, akú vlnovú dĺžku  by mal fotón vyžiarený pri  deexcitácii  atómu 
ortute. Výsledok porovnajte s pôvodným výsledkom Francka a Hertza.\\
6. Zostavte aparatúru pre stanovenie Planckovej konštanty podľa návodu.\\
7. Vykreslite závislosť energie elektrónu a frekvencie  žiarenia do grafu. Z nameraných hodnôt určite prahovú frekvenciu fotokatódy.\\
8. Z nameraných dát stanovte Planckovu konštantu a výsledok porovnajte s tabuľkovou hodnotou.
\section{Pomôcky}
fotoelektrická bunka, vysokotlaková ortuťová lampa+zdroj napätia, irisová clona, šošovka (f = 100 mm),
interferenčné filtre (578 nm, 546 nm, 436 nm, 405 nm, 365 nm), STE kľúčový spínač, zosilňovač elektrometra, STE kondenzátor (100 pF, 630 V), voltmeter a optická lavica, 2× optický jazdec (90 mm),
3× optický jazdec (120 mm), BNC adaptér, ortut'ová Franckova-Hertzova trubica, pätica pre Franckovu-Hertzovu trubicu s DIN konektorom, elektrická a pec, napájacia jednotka, teplotný senzor NiCr-Ni, dvojkanálový digitálny osciloskop GW
Instek GDS-1072B, 2× tienený kábel BNC/4 mm, 2× voltmeter, vodiče
\section{Teretický úvod}
\subsection{Fotoelektrický jav}
Fotoelektrický jav alebo fotoelektrická emisia  je experimentálne pozorovaný jav, kedy svetlo vhodnej vlnovej dĺžky pri dopade na fotoemisivný materiál, vyráža z atómov elektróny. Emitované elektróny sa nazývajú fotoelektróny. Tento jav je popísaný pomocou vzorca, ktorý dáva do súvisloti maximálnu kinetickú energiu fotoelektrónov $(E_e)$ s frekvenciou absorbovaných fotónov $(f)$ a prahovou frekvenciou fotoemisného materiálu $(f_0)$
\begin{equation}
eU_stop=hf-hf_0,
\end{equation} kde $h$ je Planckova konštanta 

\subsection{Franck-Hertzov pokus}
Pre frekvenciu fotónov emitovaných excitovaným materiálom platí vzťah 
\begin{equation}
f=E_ex/h
\end{equation}

\section{Postup merania}
\subsection{Meranie Planckovej konštanty}
Zapneme vysokotlakovú ortuťovú lampu a necháme ju zahrievať pokým nedosiahne požadovanú teplotu. Lampu, fotobunku, šošovku a clonu namontujeme na optickú lavicu pomocou vhodných optických jazdcov. Výšku ošovky a clony nastavíme tak, aby lúč svetla z ortuťovej lampy dopadal iba na fotocitlivú časť fotobunky. Umiestnime kryt na fotobunku a filter s clonou pred fotobunku. Na elektrometrický zosilovač pripojíme dva konektory, na ne kondenzátor a kľúčový spínač. Ďalej pripojíme multimeter a ten zapojíme do elektrickej siete. Zapneme multimeter a umiestnime filter pre žlté svetlo do dráhy svetelného lúča. Podržíme kľúčový spínač, kým multimeter neukáže hodnotu 0V.Meriame, pokým sa hodnota neustáli na medznom napätí $U_stop$. Postup opakujeme pre zelený, modrý, fialový a UV filter. Zmeníme intenzitu dopadajúceho svetla a merania opakujeme. 
\subsection{Franck-Hertzov pokus}
Franck-Hertzovu trubicu zasunieme do medeného valca vo vnútri elektrickej piecky. Zapneme napájaciu jednotku a skontrolujeme, že finálna teplota piecky je nastavená na 170-180 $°C$. Čakáme cca 30 minút, kým aparatúra  dosiahne požadovanú teplotu. Následne pripojíme osciloskop, na ktorom budeme pozorovať krivku, a teda závislosť prúdu $I_A$ na napätí $U_2$. Vykonáme optimalizáciu parametrov $U_1$ a $U_3$ pre požadovaný tvar krivky. Odpojíme oscilátor a k svorkám s výstupnými signálmi $U_2/10$ a $I_A$ pripojíme digitálne voltmetre. V režime MAN zmeriame 80 hodnôt pre $U_2$ v rozmedzí 0 až 30 $V$.



\section{Vypracovanie}
\section{Diskusia}
\section{Záver}
\section{Použitá literatúra}
1. Návod k úlohe Meranie kvantových vlastností atómu: Meranie Planckovej konštanty\\
\url{https://moodle-vyuka.cvut.cz/pluginfile.php/729913/mod_resource/content/2/11b_Franck-Hertz_210408.pdf}\\
2.Návod k úlohe Meranie kvantových vlastností atómu: Franck-Hertzov experiment\\
\url{https://moodle-vyuka.cvut.cz/pluginfile.php/729913/mod_resource/content/2/11b_Franck-Hertz_210408.pdf}


\begin{figure}
\centering
\includegraphics[width=0.75\linewidth]{idk.png}
\caption{\label{fig:frog}Experimentálne nastavenie na optickej lavici}
\end{figure}
\begin{figure}
\centering
\includegraphics[width=1\linewidth]{idkk.png}
\caption{\label{fig:frog}Schéma zapojenia obvodu zosilovača elektrometru}
\end{figure}

\begin{figure}
\centering
\includegraphics[width=0.75\linewidth]{Krivka.jpeg}
\caption{\label{fig:frog}Krivka zobrazená na oscilátore po optimalizácii $U_1$ a $U_3$}
\end{figure}


\end{document}
